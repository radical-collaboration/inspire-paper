\newif\ifdraft
% \drafttrue
\draftfalse

\ifdraft
 \newcommand{\jhanote}[1]{\textcolor{red}  {***SJ: #1}}
 \newcommand{\vbnote}[1]{ {\textcolor{blue} { ***VB: #1 }}}
 \newcommand{\note}[1]{ {\textcolor{blue} { ***Note: #1 }}}
 \newcommand{\mrsnote}[1]{\textcolor{olive} { ***MRS: #1 }}
 \newcommand{\jdnote}[1]{ {\textcolor{orange} { ***JD: #1 }}}
 \newcommand{\tjnote}[1]{ {\textcolor{green} { ***TJ: #1 }}}
 \newcommand{\mtnote}[1]{ {\textcolor{purple} { ***MT: #1 }}}
\else
 \newcommand{\jhanote}[1]{}
 \newcommand{\vbnote}[1]{}
 \newcommand{\note}[1]{}
 \newcommand{\mrsnote}[1]{}
 \newcommand{\jdnote}[1]{}
 \newcommand{\tjnote}[1]{}
 \newcommand{\mtnote}[1]{}
\fi


\newcommand{\cloud}{cloud\xspace}
\newcommand{\clouds}{clouds\xspace}
\newcommand{\pilot}{Pilot\xspace}
\newcommand{\pilots}{Pilots\xspace}
\newcommand{\pilotjob}{Pilot-Job\xspace}
\newcommand{\pilotjobs}{Pilot-Jobs\xspace}
\newcommand{\pilotcompute}{Pilot-Compute\xspace}
\newcommand{\pilotcomputedescription}{Pilot-Compute Description\xspace}
\newcommand{\pilotdescription}{Pilot-Description\xspace}
\newcommand{\pilotcomputes}{Pilot-Computes\xspace}
\newcommand{\pilotdata}{Pilot-Data\xspace}
\newcommand{\pilotdatadescription}{Pilot-Data Description\xspace}
\newcommand{\pilotdataservice}{Pilot-Data Service\xspace}
\newcommand{\pilotcomputeservice}{Pilot-Compute Service\xspace}
\newcommand{\computedataservice}{Compute-Data Service\xspace}
\newcommand{\computeunitdescription}{Compute-Unit Description\xspace}
\newcommand{\dataunitdescription}{Data-Unit Description\xspace}
\newcommand{\pilotmapreduce}{PilotMapReduce\xspace}
\newcommand{\mrmg}{MR-Manager\xspace}
\newcommand{\pstar}{P*\xspace}
\newcommand{\pd}{PD\xspace}
\newcommand{\pc}{PC\xspace}
\newcommand{\pcs}{PCs\xspace}
\newcommand{\pj}{PJ\xspace}
\newcommand{\pjs}{PJs\xspace}
\newcommand{\pds}{Pilot Data Service\xspace}
\newcommand{\computeunit}{Compute-Unit\xspace}
\newcommand{\computeunits}{Compute-Units\xspace}
\newcommand{\dataunit}{Data-Unit\xspace}
\newcommand{\dataunits}{Data-Units\xspace}
\newcommand{\du}{DU\xspace}
\newcommand{\dus}{DUs\xspace}
\newcommand{\dud}{DUD\xspace}
\newcommand{\cu}{CU\xspace}
\newcommand{\cus}{CUs\xspace}
\newcommand{\cud}{CUD\xspace}
\newcommand{\su}{SU\xspace}
\newcommand{\sus}{SUs\xspace}
\newcommand{\schedulableunit}{Schedulable Unit\xspace}
\newcommand{\schedulableunits}{Schedulable Units\xspace}
\newcommand{\cc}{c\&c\xspace}
\newcommand{\CC}{C\&C\xspace}
\newcommand{\up}{\vspace*{-1em}}
\newcommand{\upp}{\vspace*{-0.5em}}
\newcommand{\numrep}{8 }
\newcommand{\samplenum}{4 }
\newcommand{\tmax}{$T_{max}$ }
\newcommand{\tc}{$T_{C}$ }
\newcommand{\tcnsp}{$T_{C}$}
\newcommand{\bj}{BigJob\xspace}
\newcommand{\irods}{iRODS\xspace}
\newcommand{\entk}{EnTK\xspace}
\newcommand{\rp}{RADICAL-Pilot\xspace}
\newcommand{\ee}{expanded ensemble\xspace}
\newcommand{\msm}{adaptive msm\xspace}

\newcommand{\I}[1]{\textit{#1}\xspace}
\newcommand{\B}[1]{\textbf{#1}\xspace}
\newcommand{\T}[1]{\texttt{#1}\xspace}
%\newcommand{\C}[1]{\textsc{#1}\xspace}

\newcommand{\mr}[1]{\multirow{2}{*}{#1}}%
\newcommand{\mc}[2]{\multicolumn{#1}{l}{#2}}

 \lstset{ backgroundcolor=\color{white},
 basicstyle=\ttfamily\scriptsize, % the size of the fonts that are used for the code 
 breakatwhitespace=false, % sets if automatic breaks should only happen at whitespace 
 breaklines=true, % sets automatic line breaking 
 captionpos=b, % sets the caption-position to bottom 
 commentstyle=\color{mygreen}, % comment style 
 escapeinside={\%*}{*)}, % if you want to add LaTeX within your code 
 extendedchars=true, % lets you use non-ASCII characters; for 8-bits encreadreading twitter ing twitter odings only, does not work with UTF-8
 frame=single, % adds a frame around the code 
 keepspaces=true, % keeps spaces in text, useful for keeping indentation of code (possibly needs columns=flexible)  
 keywordstyle=\color{blue}, % keyword style 
 language=Python, % the language of the code 
 numbers=none, % where to put the line-numbers; possible values are (none, left, right) 
 numbersep=5pt, % how far the line-numbers are from the code 
 numberstyle=\tiny\color{gray}, % the style that is used for the line-numbers 
 rulecolor=\color{white}, % if not set, the frame-color may be changed on line-breaks within not-black text (e.g. comments (green here)) 
 showspaces=false, % show spaces everywhere adding particular underscores; it overrides 'showstringspaces' 
 showstringspaces=false, % underline spaces within strings only 
 showtabs=false, % show tabs within strings adding particular underscores 
 stepnumber=2, % the step between two line-numbers. If it's 1, each line will be numbered 
 stringstyle=\color{mauve}, % string literal style 
 tabsize=2, % sets default tabsize to 2 spaces 
}

%  \setlength{\parskip}{0.05ex} % 1ex plus 0.5ex minus 0.2ex}
%  \setlength{\parsep}{0pt}
%  %\setlength{\headsep}{0pt}
%  \setlength{\topskip}{0pt}
%  \setlength{\topmargin}{0pt}
%  %\setlength{\topsep}{0pt}
%  \setlength{\partopsep}{0pt}

% This is now the recommended way for checking for PDFLaTeX:


\ifpdf
\DeclareGraphicsExtensions{.pdf, .jpg, .tif}
\else
\DeclareGraphicsExtensions{.ps,  .eps, .jpg}
\fi

\tolerance=1000
\hyphenpenalty=10

\pdfsuppresswarningpagegroup=1


\let\origdescription\description
\renewenvironment{description}{
  \setlength{\leftmargini}{0em}
  \origdescription
  \setlength{\itemindent}{0em}
  \setlength{\labelsep}{\textwidth}
}
{\endlist}
